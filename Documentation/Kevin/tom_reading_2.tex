\documentclass[11pt,a4paper,english]{../uvamath}
\usepackage[english]{babel}

\usepackage{amsmath, amsfonts, amssymb, a4wide, fancyhdr, lineno, graphicx, epsfig, soul, color, hyperref}
\usepackage[square, numbers]{natbib}

% Running line numbers:
\linenumbers
% Number only every 5:th line:
\modulolinenumbers[5]

%Nodig om een bibliography midden in het artikel te zetten, ipv aan het einde zoals eigenlijk gebruikelijk is
\renewcommand{\bibsection}{}

% TODO command
\newcommand{\todo}[1]{
    \hl{#1}
}

% The things that should be filled in by each group, depending on their situation, are written in a todo command, \todo{like this text}. All text in normal the normal font, is applicable for any group. However, everyone is free to adapt any text, and it is even suggested to look at all text critically and make changes if needed.


\newcommand{\projectname}{\todo{Project Name}\ }


\newcommand{\aanpassen}[1]{ {\sethlcolor{green} \hl{#1}} }

\title{Reading Assignment week 2}
%Variables
\newcommand{\TitelAbbr}{}
\newcommand{\Version}{0.1}



\what{}
\supervisors{}
\author{Kevin van den Bekerom}


\begin{document}

\maketitle

\clearpage

\chapter*{Week 2. Getting rich and useful data}

\section*{2.1}

\begin{enumerate}
	\item "There are no right or wrong answers." Reassure interviewee who thinks himself non-intelligent (on the subject).  \cite{nisha}
	\item "Don't you", "wouldn't you", "shouldn't you" followed by leading question. \cite{nisha}.
	\item Smile (for shy people) \cite{nisha}.
	\item Pause (After asking question, so interviewee feels no rush in formulating their answer). \cite{nisha}
	\item Reformulate the answer of the interviewee, possible revealing structures he/she was unaware of. If the interviewer gives a wrong summary (his mental model differs from the interviewee), then this leads to an invitation for the interviewee to elaborate further.  \cite{apprenticing}.
\end{enumerate}

\section*{2.2}

\begin{enumerate}
	\item Interrupting the interviewee. Doing so will  \cite{medical}.
	\item A long pause, which makes the interviewee uneasy \cite{nisha}.
	\item \textit{Huh}?  \cite{apprenticing}.
	\item Half-focused on the interviewee. \cite{medical}.
	\item Be really energetic. \cite{nisha} 
\end{enumerate}

\section*{2.3}

\subsection*{Positive cues}
\textbf{Summarizing what the interviewee said}: The interviewee gave a long speech about different usage of Blackboard among teacher. Tom summarized his view concise. The interviewee had a different mental model and elaborated on another important aspect (he was not against flexibility). Tom does this multiple times, and the interviewee is triggered to elaborate.

\textbf{Ask about specific events (5)} Tom asked about the the duration the interviewee used Blackboard yesterday. This triggered concrete episodes, where the interviewee could give a somewhat precise answer. While the interviewee was triggered, Tom could have asked to make things specific and ask exactly how much minutes the interviewee actively used Blackboard.

\textbf{Pause}

\subsection*{Negative cues}
(After sharing the ones my partner found)
\textbf{Hasty reassurances and premature advising/interrupting (2/3)} Tom sometimes cut off the interviewee when he was forming his next sentence. An example is the first question. 

\textbf{Talking in the abstract (5)} The question about what the interviewee would do if the UvA decided to discontinue BlackBoard entirely and not replace it, which was very abstractly framed, we got a very abstract, and not very usable, answer. Framing the question differently, or work towards it with some specific questions could possibly prevent an abstract answer. For example asking what other tools the interviewee uses now, or has knowledge of, priming him for tools that could potentially replace BlackBoard and then asking the hypothetical question what he would use if BlackBoard would not exist tomorrow.

\section*{2.4}



So techniques that make the interviewee think about a specific event or a specific action that has happened in the past will yield more reliable memories than thinking in abstract terms about an abstract event or task.

\section*{2.5}

Unfortunately our sample video is too short to draw any conclusions.

\chapter{References}

\begin{thebibliography}{9}	
	\bibitem{medical} 
	Preparing medical students to become attentive listeners, \\
	\emph{Dr J. Donald Boudreau, Eric Cassell \& Abraham Fuks.}
	
	\bibitem{apprenticing}
	Apprenticing With the Customer \\
	\emph{Hugh R. Beyer \& Karen Holtzblatt}
	
	\bibitem{kahneman}
	Thinking, Fast and Slow \\
	\emph{Daniel Kahneman}
	
	\bibitem{nisha}
	Requirements Elicitation Technique: Improving the Interview Technique \\
	emph{} Nisha Jacob}\\
    (10629505, http://dare.uva.nl/en/scriptie/521450) 
\end{thebibliography}

\begin{thebibliography}{9}
	
\end{thebibliography}


\appendix


\end{document}
