\documentclass[11pt,a4paper,english]{uvamath}
\usepackage[english]{babel}

\usepackage{amsmath, amsfonts, amssymb, a4wide, fancyhdr, lineno, graphicx, epsfig, soul, color, hyperref}
\usepackage[square, numbers]{natbib}

% Running line numbers:
\linenumbers
% Number only every 5:th line:
\modulolinenumbers[5]

%Nodig om een bibliography midden in het artikel te zetten, ipv aan het einde zoals eigenlijk gebruikelijk is
\renewcommand{\bibsection}{}

% TODO command
\newcommand{\todo}[1]{
    \hl{#1}
}

% The things that should be filled in by each group, depending on their situation, are written in a todo command, \todo{like this text}. All text in normal the normal font, is applicable for any group. However, everyone is free to adapt any text, and it is even suggested to look at all text critically and make changes if needed.

% Project specific commands
\author{Tom van Duist \& Kevin van den Bekerom}

\newcommand{\projectname}{\todo{Project Name}\ }


\newcommand{\aanpassen}[1]{ {\sethlcolor{green} \hl{#1}} }

\title{Blackboard log}
\what{}
\supervisors{}
\author{Kevin van den Bekerom}


\begin{document}

\maketitle



\clearpage

\chapter{Personal BlackBoard use}
I use Blackboard because it is imposed by the teachers. Whenever otherwise possible, I avoid Blackboard. A good example is the Software Testing course, where the teacher had a personalized website with all relevant course material. My previous experiences consisted mostly of using course websites, as designed by the teachers themselves (Bachelor Software Science at Eindhoven, University of Technology). I used Blackboard to access information required to pass the course, which I cannot find anywhere else (slides, assignments, articles, ...). Whenever required, I use Blackboard to hand in assignments. I have been using Blackboard for two months now. 


\section{BlackBoard log}

\textbf{Download files:}
\begin{enumerate}
	\item Log onto Blackboard by typing "uva blackboa.." into search bar. Google automatically displays link.
	\item Whatever course I use most is the first displayed option. If this is not the course I need I still use this option. I subsequently select option correct course from dropdown menu at the top (recently visited).
	\item Select option "Course materials" in left menu.
	\item Look for any new files and download them to my local pc.
	\item Repeat step 2 and 3 for the sections 'Assignments', 'Course Information' and 'Announcements'.
	\item Synchronize files using Dropbox.
\end{enumerate}

Note: If some form of information is presented in Blackboard as plain text (no download link) I will consistently visit Blackboard to access that information.\\

\textbf{Hand in assignment:}
\begin{enumerate}
	\item Go to relevant course.
	\item Click on 'Assignments'.
	\item Click on relevant assignment title.
	\item Click on continue in the top right corner.
	\item Find the upload or browse button and upload the file.
	\item Click submit (this will automatically show the created submission).
	\item Double check if it is the correct file by downloading the submission by clicking the download button next to the file.
\end{enumerate}

\pagebreak[4]

\textbf{Read announcements:}\\
\textbf{Trigger}: Whenever teacher or classmate notifies me that new information is available on Blackboard.
\begin{enumerate}
	\item Go to relevant course.
	\item Click on 'Announcements'.
	\item Read new announcements.
\end{enumerate}

\textbf{View grades:}
A teacher for software testing (Vadim) showed me that I could access grades on Blackboard, and how to access them.


\section{Observations}

\begin{itemize}
	\item I do not use any notifications about Blackboard. I just log in whenever I need something for an assignment, or need to check where a certain meeting is held.
	\item Discovering that the "continue" button takes one to the hand-in screen for assignments was unclear the first time. Also: sometimes Blackboard immediately displays this screen when clicking on an assignment, sometimes it does not.
	\item There is no clear organization of information. For the Software Architecture course I had to find the slides between all the papers, which considerably slowed down my retrieval time while at the same time annoyed me.
	\item Old announcements will be pushed to the bottom when a new announcement is added. This makes finding back relevant general information harder and more annoying. 
\end{itemize}


\section{Conclusions}

\begin{itemize}
	\item The process of handing in an assignment is confusing.
	\item I go back to Blackboard to lookup information that requires a bit of effort to save locally. 
	\item Easy way to make folder structure to organize course information (map for lecture notes, papers, etc.) will be a good feature.
	\item General course information that will stick to the top, or be available in a predetermined place is a convenient feature.
	\item Button to go back to general screen from course screen is missing (or otherwise not findable).
\end{itemize}


\begin{thebibliography}{9}
	
\end{thebibliography}


\appendix


\end{document}
