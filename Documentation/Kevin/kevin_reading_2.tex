\documentclass[11pt,a4paper,english]{uvamath}
\usepackage[english]{babel}

\usepackage{amsmath, amsfonts, amssymb, a4wide, fancyhdr, lineno, graphicx, epsfig, soul, color, hyperref}
\usepackage[square, numbers]{natbib}

% Running line numbers:
\linenumbers
% Number only every 5:th line:
\modulolinenumbers[5]

%Nodig om een bibliography midden in het artikel te zetten, ipv aan het einde zoals eigenlijk gebruikelijk is
\renewcommand{\bibsection}{}

% TODO command
\newcommand{\todo}[1]{
    \hl{#1}
}

% The things that should be filled in by each group, depending on their situation, are written in a todo command, \todo{like this text}. All text in normal the normal font, is applicable for any group. However, everyone is free to adapt any text, and it is even suggested to look at all text critically and make changes if needed.

% Project specific commands
\author{Tom van Duist \& Kevin van den Bekerom}

\newcommand{\projectname}{\todo{Project Name}\ }


\newcommand{\aanpassen}[1]{ {\sethlcolor{green} \hl{#1}} }

\title{Reading Assignment week 2}
%Variables
\newcommand{\TitelAbbr}{}
\newcommand{\Version}{0.1}



\what{}
\supervisors{}
\author{Kevin van den Bekerom}


\begin{document}

\maketitle

\clearpage

\chapter*{Week 2. Getting rich and useful data}

\section*{2.1}

\begin{enumerate}
	\item "There are no right or wrong answers." Reassure interviewee who thinks himself non-intelligent (on the subject).  \cite{nisha}
	\item "Don't you", "wouldn't you", "shouldn't you" followed by leading question. \cite{nisha}.
	\item Smile (for shy people) \cite{nisha}.
	\item Pause (After asking question, so interviewee feels no rush in formulating their answer). \cite{nisha}
	\item Reformulate the answer of the interviewee, possible revealing structures he/she was unaware of. If the interviewer gives a wrong summary (his mental model differs from the interviewee), then this leads to an invitation for the interviewee to elaborate further.  \cite{apprenticing}.
\end{enumerate}

\section*{2.2}

\begin{enumerate}
	\item Interrupting the interviewee. \cite{medical}.
	\item A long pause, which makes the interviewee uneasy \cite{nisha}.
	\item \textit{Huh}?  \cite{apprenticing}.
	\item Half-focused on the interviewee. \cite{medical}.
	\item Be really energetic. \cite{nisha} 
\end{enumerate}

\section*{2.3}
(Also based on cues my partner found)
Based on interview with another Master student.

\subsection*{Positive cues}
\textbf{Summarizing what the interviewee said}: The interviewee gave a long speech about different usage of Blackboard among teacher. Tom summarized his view concise. The interviewee had a different mental model and elaborated on another important aspect (he was not against flexibility). Tom does this multiple times, and the interviewee is triggered to elaborate.

\textbf{Ask about specific events (5)} Tom asked about the the duration the interviewee used Blackboard yesterday. This triggered concrete episodes, where the interviewee could give a somewhat precise answer. While the interviewee was triggered, Tom could have asked to make things specific and ask exactly how much minutes the interviewee actively used Blackboard.

\textbf{Pause after question}: Tom always gave the interviewee enough time to formulate an answer. The interviewee was evidently comfortable in taking the time to come up with elaborate answers.

\subsection*{Negative cues}

\textbf{Hasty reassurances and premature advising/interrupting} Tom sometimes cut off the interviewee when he was forming his next sentence. An example is the first question. 

\section*{2.4}
When asking abstract questions, we respond by giving a summary of what we actually do, leaving out steps we take for granted. However, when asked about a concrete event in the (recent) past, you give answers that trigger memories in your brain about the steps you left out in the abstract question. Reporting one specific step will trigger a memory about the next step, etc. \cite{apprenticing}

Priming is a technique that triggers certain memories before asking the real question. The interviewer might suggest Blackboard is a bad system by asking "Besides the ugly interface, what do you think is bad about Blackboard." The interviewee is triggered by other programs with bad interfaces, and the word "bad" triggers negative things in general, so the interviewee has little effort retrieving bad facts about Blackboard. \cite{kahneman}.

Interview techniques that focus on recent memories are more effective then abstract questions. The interviewer should be careful for the priming effect, since it results in unreliable information.

\section*{2.5}

Sample video is of insufficient length to draw meaningful conclusions.

\chapter{References}

\begin{thebibliography}{9}	
	\bibitem{medical} 
	Preparing medical students to become attentive listeners, \\
	\emph{Dr J. Donald Boudreau, Eric Cassell \& Abraham Fuks.}
	
	\bibitem{apprenticing}
	Apprenticing With the Customer \\
	\emph{Hugh R. Beyer \& Karen Holtzblatt}
	
	\bibitem{kahneman}
	Thinking, Fast and Slow \\
	\emph{Daniel Kahneman}
	
	\bibitem{nisha}
	Requirements Elicitation Technique: Improving the Interview Technique \\
	\emph{ Nisha Jacob}\\
    (10629505, http://dare.uva.nl/en/scriptie/521450) 
\end{thebibliography}

\begin{thebibliography}{9}
	
\end{thebibliography}


\appendix


\end{document}
