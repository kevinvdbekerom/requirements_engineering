\documentclass[11pt,a4paper,english]{uvamath}
\usepackage[english]{babel}

\usepackage{amsmath, amsfonts, amssymb, a4wide, fancyhdr, lineno, graphicx, epsfig, soul, color, hyperref}
\usepackage[square, numbers]{natbib}

% Running line numbers:
\linenumbers
% Number only every 5:th line:
\modulolinenumbers[5]

%Nodig om een bibliography midden in het artikel te zetten, ipv aan het einde zoals eigenlijk gebruikelijk is
\renewcommand{\bibsection}{}

% TODO command
\newcommand{\todo}[1]{
    \hl{#1}
}

% The things that should be filled in by each group, depending on their situation, are written in a todo command, \todo{like this text}. All text in normal the normal font, is applicable for any group. However, everyone is free to adapt any text, and it is even suggested to look at all text critically and make changes if needed.

% Project specific commands
\author{Tom van Duist \& Kevin van den Bekerom}

\newcommand{\projectname}{\todo{Project Name}\ }


\newcommand{\aanpassen}[1]{ {\sethlcolor{green} \hl{#1}} }

\title{Reading assignment week 3}
%Variables
\newcommand{\TitelAbbr}{}
\newcommand{\Version}{0.1}



\what{}
\supervisors{}
\author{Kevin van den Bekerom}


\begin{document}

\maketitle

\clearpage


\chapter*{Reading assignments week 3}

\section*{3.1}
First I present my own perception, then the perception of 4 other Master students. After that I quickly describe commonalities, after which I describe differences. Whenever applicable, these differences are explained using the Natural Decision Making \cite{ndm}, and Crisis thinking as in \cite{crisis}.

\subsection*{My perception}
\textbf{Assignment}: Course assignment for the first four weeks, i.e. finding requirements for the successor of Blackboard.
\textbf{Learning goals}: Course goals, i.e. what do we need to have learned after completing the requirements engineering course.

When I received the assignment it was very vague. I based what I had to do on the assignments of week 1. My vision was formed by the literature I read, which narrowed my view.
The second week I learned - during the Thursday session - what I did wrong. My view of the assignment now (week 3) is that it was intentionally vague in order for me to make
the same mistakes as guys as Richard, and other Requirements Engineers made. It really stresses the fact HOW you fail, which in turn solidifies the knowledge how to do it right.
Hans could have said what to do exactly, or what NOT to do, but then I would not have had this failing experience. I would end the course with the belief that I am a good requirements
engineer who doesn't make the mistakes of those we went before him. As I now know, this would be extremely naive. \\

\textbf{Conclusion}: The assignment is made vague on purpose to assess if students really understood how to do requirements engineering the right way. Most often this will result in them having a failure experience,
which is very learn-full. 

\subsection{perception student 1}
My learning goals are to acquire knowledge and getting a degree. This course focusses in my opinion
very much on learning by self-reflection. By being confronted with your own mistakes and having
discussions within or outside your own team, you will be able to learn more instead of the teacher
telling us what his solution is to requirements engineering. For this assignment we need to use many
techniques for decision making which help us to acquire a lot of knowledge in the field of
requirements engineering.


\subsection{perception student 2}
I believe that this assignment helps me achieve my goals because it really lets me learn from practical
experience rather than only studying the theory. By applying my knowledge in practice I myself can
review my work after and see what I did right and wrong. And by doing practical work I can more
quickly learn from my mistakes and thus quickly recover from them. I also believe that due to the
limited time we have to make the all assignments (because of the deadline on Friday) we need to use
our time really efficiently. And due to the short deadline, you could say there is a small crisis situation
going on because of all the work which has to be done relatively quickly. This requires that we really
think about the things we want to achieve to satisfy the core of the assignments and not waste more
time on “gold plating" our assignments.

\subsection{perception student 3}
Before taking on this course I was always curious about the techniques a requirement engineer
should undertake in order to find out the needs for a software system. I knew conducting interviews
was one of the ways to go for, but I never would have realized that there existed a lot of possibilities
to gather information from users. At first we were not good at interviewing. The amount of
information we got from our interviewees in the first week was very small and they were basically all
the information we already knew, but as we started to read the materials on the weekly basis and
also participating in the lectures and workshops and using the feedback form our teachers we began
to learn quite a lot ways for acquiring useful information from the users. For example we started to
get familiar with how the mind works and how the information provided from users may be either
incomplete or not even truthful. We began to learn how to use cognitive aids and some cues in order
to help the interviewees to provide information as much as possible by triggering their minds. One of
the main things that helped us achieve our learning goals was using all these information in practice
by conducting interviews.

\subsection{perception student 4}
My interpretation of the assignment, or maybe I should say my guess, is we are supposed to try to find some useful requirements for a successor of BB, but have to do so while trying out all kinds of suitable and less suitable techniques. I think learning to apply these techniques is the most important learning goal for this module, so applying them should be usefull, but I think having to find usefull requirements too, without getting to choose an appropriate technique to apply, forces us to use the techniques in such weird ways we don't actually learn to use them very well.

\subsection{Commonalities}
\begin{itemize}
	\item Practical experience: All students believe the course aims at practical experience with which to achieve the learning goals. 
	\item Purposely vague course start: The start of the course should be viewed as a crisis. Most of the students just started doing the weekly assignments in the limited time, trying out techniques that were at hand (papers, lectures), since they could not rely on past experiences. The RPD model \cite{ndm} explains that we base our decision on whatever knowledge comes most easily to mind, being the lectures and papers just read. In subsequent weeks we had more knowledge, which broadens our view, making us more comfortable to have a response to more crisis situations. This explains the common trend that students have more faith in the course at this stage then in the beginning.  
\end{itemize}

\subsection{Differences}
	\begin{itemize}
		\item Practical experience works vs it doesn't: The assignment and the shortage of time to complete it can be viewed as a crisis. Some students believe applying novel techniques for gathering information (also requirements) helps remedy this crisis, while others think it makes it worse. As Weick explained in his paper both can be right. We make sense of a novel situation by trying out actions. These actions become part of the crisis. If we chose the wrong action, we make sense of a situation in which this wrong action is now incorporated, making the crisis worse. We might then conclude that the action itself does not work.
	\end{itemize}


\section*{3.2}
I list in chronological order the tasks I undertook previous week. Whenever applicable, I will describe if the task did not come to mind in that precise order. For some tasks I elaborate on my rationale to start that tasks, using the NDM and RPM models. \cite{ndm}. \\

\begin{enumerate}
	\item Reading Thinking Fast and Slow. Started already on monday, since I knew - from past experience - that I most probably needed the information in this book during my interviews. I also had to read it for the reading test - presumably this week - so it wouldn't hurt.
	\item Reading the assignments on Blackboard. Did this just after the lecture. Was the first thing that came to mind and I went with it. I relied on my intuition. Interestingly, I recalled this task only after remembering all other tasks. I believe that my reason for doing this task - being the most obvious task to do - did not leave a big footprint in my memory. System 1 decided for me \cite{kahneman}, almost unconsciously. 
	\item Read papers. For the assignments we were required to use knowledge from the papers. 
	\item Interview preparation (wednesday evening). After reading most relevant - not all - papers first. Effective use of time.
	\item Interviews on thursday 9-11. We had to present our findings on the meeting at 11. We wanted to read the papers first, so this was the only timeslot to fit the interviews. We tried to interview a teacher. While we got some interesting - not previously encountered - answers, we spent a lot of time finding a teacher. Time not spent effectively. Next time maybe arrange meeting with teacher beforehand.
	\item Group assignments. \begin{enumerate}
		\item Research + reading 13:30 - 16:00. I do not recall why we chose research before structuring data. It might be that research was a more concrete assignment, for which I could visualize myself having success. This would explain why this assignment was chosen over the more abstract \emph{structure data} assignment.
		\item structuring data. 16:00 - 17:00. We really timeboxed this assignment. 
	\end{enumerate}
	\item Put knowledge in document and portfolio. Friday 22:00 - 23:45. 
\end{enumerate}

\clearpage
\section*{3.3}
\begin{enumerate}
	\item Share course announcements. (Achieve)
	\item If student passed course then always no access to course announcements.
	\item Distribute course information. (Achieve)
	\item If student passed course then always access to course information.
	\item Distribute course materials. (Achieve)
	\item If student has access to course then always course materials readable. 
	\item If student passed course, then always course materials readable. (Remain access to course slides etc from previous courses.)
	\item Only students enrolled in course X have access to course announcements. (Integrity)
	\item Only students (at some point) enrolled in course X can access course materials and course information.
	\item Maintain course grades. 
	\item If student started course then eventually receives grade.
	\item Course information should be accessible within 5 seconds after logging into system. (Performance, Usability)
\end{enumerate}


\chapter{References}

\begin{thebibliography}{9}

	\bibitem{ndm}
	Naturalistic Decision Making, \\
	\emph{Gary Klein, Klein Associates, Division of ARA, Fairborn, Ohio.}
	
	\bibitem{crisis}
	Enacted Sensemaking in Crisis Situations, \\
	\emph{Karl E. Weick, University of Michigan, USA.}
	
\end{thebibliography}


\appendix


\end{document}
