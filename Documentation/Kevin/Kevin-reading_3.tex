\documentclass[11pt,a4paper,english]{../uvamath}
\usepackage[english]{babel}

\usepackage{amsmath, amsfonts, amssymb, a4wide, fancyhdr, lineno, graphicx, epsfig, soul, color, hyperref}
\usepackage[square, numbers]{natbib}

% Running line numbers:
\linenumbers
% Number only every 5:th line:
\modulolinenumbers[5]

%Nodig om een bibliography midden in het artikel te zetten, ipv aan het einde zoals eigenlijk gebruikelijk is
\renewcommand{\bibsection}{}

% TODO command
\newcommand{\todo}[1]{
    \hl{#1}
}

% The things that should be filled in by each group, depending on their situation, are written in a todo command, \todo{like this text}. All text in normal the normal font, is applicable for any group. However, everyone is free to adapt any text, and it is even suggested to look at all text critically and make changes if needed.


\newcommand{\projectname}{\todo{Project Name}\ }


\newcommand{\aanpassen}[1]{ {\sethlcolor{green} \hl{#1}} }

\title{Reading assignment week 3}
%Variables
\newcommand{\TitelAbbr}{}
\newcommand{\Version}{0.1}



\what{}
\supervisors{}
\author{Kevin van den Bekerom}


\begin{document}

\maketitle

\clearpage


\chapter*{Reading assignments week 3}

\section*{3.1}

\todo{explain differences in course assignment perception in accordance with the Weick and Klein papers}

\textbf{Assignment}: Course assignment for the first four weeks, i.e. finding requirements for the successor of Blackboard.
\textbf{Learning goals}: Course goals, i.e. what do we need to have learned after completing the requirements engineering course.

When I received the assignment it was very vague. I based what I had to do on the assignments of week 1. My vision was formed by the literature I read, which narrowed my view.
The second week I learned - during the Thursday session - what I did wrong. My view of the assignment now (week 3) is that it was intentionally vague in order for me to make
the same mistakes as guys as Richard, and other Requirements Engineers made. It really stresses the fact HOW you fail, which in turn solidifies the knowledge how to do it right.
Hans could have said what to do exactly, or what NOT to do, but then I would not have had this failing experience. I would end the course with the belief that I am a good requirements
engineer who doesn't make the mistakes of those we went before him. As I now know, this would be extremely naive. \\

\textbf{Conclusion}: The assignment is made vague on purpose to assess if students really understood how to do requirements engineering the right way. Most often this will result in them having a failure experience,
which is very learn-full. 


\section*{3.2}
I list in chronological order the tasks I undertook previous week. Whenever applicable, I will describe if the task did not come to mind in that precise order. For some tasks I elaborate on my rationale to start that tasks, using the NDM and RPM models. \cite{ndm}. \\

\begin{enumerate}
	\item Reading Thinking Fast and Slow. Started already on monday, since I knew - from past experience - that I most probably needed the information in this book during my interviews. I also had to read it for the reading test - presumably this week - so it wouldn't hurt.
	\item Reading the assignments on Blackboard. Did this just after the lecture. Was the first thing that came to mind and I went with it. I relied on my intuition. Interestingly, I recalled this task only after remembering all other tasks. I believe that my reason for doing this task - being the most obvious task to do - did not leave a big footprint in my memory. System 1 decided for me \cite{kahneman}, almost unconsciously. 
	\item Read papers. For the assignments we were required to use knowledge from the papers. 
	\item Interview preparation (wednesday evening). After reading most relevant - not all - papers first. Effective use of time.
	\item Interviews on thursday 9-11. We had to present our findings on the meeting at 11. We wanted to read the papers first, so this was the only timeslot to fit the interviews. We tried to interview a teacher. While we got some interesting - not previously encountered - answers, we spent a lot of time finding a teacher. Time not spent effectively. Next time maybe arrange meeting with teacher beforehand.
	\item Group assignments. \begin{enumerate}
		\item Research + reading 13:30 - 16:00. I do not recall why we chose research before structuring data. It might be that research was a more concrete assignment, for which I could visualize myself having success. This would explain why this assignment was chosen over the more abstract \emph{structure data} assignment.
		\item structuring data. 16:00 - 17:00. We really timeboxed this assignment. 
	\end{enumerate}
	\item Put knowledge in document and portfolio. Friday 22:00 - 23:45. 
\end{enumerate}

\clearpage
\section*{3.3}



\chapter{References}

\begin{thebibliography}{9}
	
\end{thebibliography}


\appendix


\end{document}
