\documentclass[11pt,a4paper,english]{uvamath}
\usepackage[english]{babel}

\usepackage{amsmath, amsfonts, amssymb, a4wide, fancyhdr, lineno, graphicx, epsfig, soul, color, hyperref}
\usepackage[square, numbers]{natbib}

% Running line numbers:
\linenumbers
% Number only every 5:th line:
\modulolinenumbers[5]

%Nodig om een bibliography midden in het artikel te zetten, ipv aan het einde zoals eigenlijk gebruikelijk is
\renewcommand{\bibsection}{}

% TODO command
\newcommand{\todo}[1]{
    \hl{#1}
}

% The things that should be filled in by each group, depending on their situation, are written in a todo command, \todo{like this text}. All text in normal the normal font, is applicable for any group. However, everyone is free to adapt any text, and it is even suggested to look at all text critically and make changes if needed.

% Project specific commands
\author{Tom van Duist \& Kevin van den Bekerom}

\newcommand{\projectname}{\todo{Project Name}\ }


\newcommand{\aanpassen}[1]{ {\sethlcolor{green} \hl{#1}} }

\title{Reading Assignment week 2}
%Variables
\newcommand{\TitelAbbr}{}
\newcommand{\Version}{0.1}



\what{}
\supervisors{}
\author{Tom van Duist}


\begin{document}

\maketitle

\clearpage

\chapter*{Week 2. Getting rich and useful data}

\section*{2.1}

\begin{enumerate}
	\item Speak slowly and softly using regular pauses that seem deliberate \cite{medical}.
	\item When you speak as if you were the interviewee, act as if you are uneducated on the subject \cite{medical}.
	\item In you speech (when responding to an answer) show that you are perplexed and unsure \cite{medical}.
	\item Repeat (summarize) what the interviewee said in your own words and interpretation, preferably revealing a possible structure, and ask for confirmation \cite{apprenticing}.
	\item Ask about specific events that have occurred \cite{apprenticing}.
\end{enumerate}

\section*{2.2}

\begin{enumerate}
	\item The act of (seeming to be) half, distracted or selective listening \cite{medical}.
	\item Interrupting the interviewee, this could be a cue that you're not interested \cite{medical}.
	\item Hasty reassurances and premature advising \cite{medical}. 
	\item Discomfort due to silence or long pauses \cite{medical}.
	\item Questions that make the interviewee talk (and think) in the abstract \cite{apprenticing}.
\end{enumerate}

\section*{2.3}

\subsection*{Positive cues}
\textbf{Deliberate pauses (1)} They are performed, but I find it hard to judge if they seem deliberate to the interviewee or not, to me they do but im biased towards this. It does seem to yield results but I cannot prove this from a single interview.

\textbf{Ask about specific events (5)} Kevin asked about a specific event that might have occurred yesterday, but when this did had not occurred he started talking in the abstract about last week (looking up to the ceiling confirmed this \cite{apprenticing}). Kevin could've mended the situation by asking about a specific event that occurred during that week.

\subsection*{Negative cues}
\textbf{Hasty reassurances and premature advising/interrupting (2/3)} Kevin responding a couple of times with \emph{'yes'} while the interviewee was talking, this could give the impression that Kevin already understood his point. On some occasions the interviewee stopped elaborating his point further.

\textbf{Talking in the abstract (5)} The question about what the interviewee would do if the UvA decided to discontinue BlackBoard entirely and not replace it, which was very abstractly framed, we got a very abstract, and not very usable, answer. Framing the question differently, or work towards it with some specific questions could possibly prevent an abstract answer. For example asking what other tools the interviewee uses now, or has knowledge of, priming him for tools that could potentially replace BlackBoard and then asking the hypothetical question what he would use if BlackBoard would not exist tomorrow.

\section*{2.4}

Some techniques yield abstract answers and when people are thinking in the abstract a lot of the context from the event(s) is lost which will not give a very detailed answer and important steps might even be overlooked \cite{apprenticing} or even worse, as Kahneman showed the mind can make up things to fill in the gaps and this could even happen without the interviewer or even the interviewee himself noticing as long as the story stays consistent and coherent \cite{kahneman}.

When people are talking about a specific event that has occurred in the past that specific event can trigger other memories that are connected to this event. Thus when people are talking in specific terms about specific events that have occurred the relevant and detailed memories of those events will come to mind \cite{apprenticing}\cite{kahneman}.

So techniques that make the interviewee think about a specific event or a specific action that has happened in the past will yield more reliable memories than thinking in abstract terms about an abstract event or task.

\section*{2.5}

Unfortunately our sample video is too short to draw any conclusions.

\chapter{References}

\begin{thebibliography}{9}	
	\bibitem{medical} 
	Preparing medical students to become attentive listeners, \\
	\emph{Dr J. Donald Boudreau, Eric Cassell \& Abraham Fuks.}
	
	\bibitem{apprenticing}
	Apprenticing With the Customer \\
	\emph{Hugh R. Beyer \& Karen Holtzblatt}
	
	\bibitem{kahneman}
	Thinking, Fast and Slow \\
	\emph{Daniel Kahneman}
\end{thebibliography}

\begin{thebibliography}{9}
	
\end{thebibliography}


\appendix


\end{document}
