\documentclass[11pt,a4paper,english]{../uvamath}
\usepackage[english]{babel}

\usepackage{amsmath, amsfonts, amssymb, a4wide, fancyhdr, lineno, graphicx, epsfig, soul, color, hyperref}
\usepackage[square, numbers]{natbib}

% Running line numbers:
\linenumbers
% Number only every 5:th line:
\modulolinenumbers[5]

%Nodig om een bibliography midden in het artikel te zetten, ipv aan het einde zoals eigenlijk gebruikelijk is
\renewcommand{\bibsection}{}

% TODO command
\newcommand{\todo}[1]{
    \hl{#1}
}

% The things that should be filled in by each group, depending on their situation, are written in a todo command, \todo{like this text}. All text in normal the normal font, is applicable for any group. However, everyone is free to adapt any text, and it is even suggested to look at all text critically and make changes if needed.


\newcommand{\projectname}{\todo{Project Name}\ }


\newcommand{\aanpassen}[1]{ {\sethlcolor{green} \hl{#1}} }

\title{Reading assignment week 3}
%Variables
\newcommand{\TitelAbbr}{}
\newcommand{\Version}{0.1}



\what{}
\supervisors{}
\author{Tom van Duist}


\begin{document}

\maketitle

\clearpage


\chapter*{Reading assignments week 3}

\section*{3.1}
Talking to different students there were not a lot of differences in their perception with regards to the assignment and how that perception relates to their study goals. Possibly everybody formed the same ideas by talking to each other. This could also be explained through the recognition-primed decision (RPD) model as described by Klein \cite{klein} and about Sensemaking by Weick \cite{weick}.

Because the onset of this course is very different than other courses we have experienced, we cannot draw from personal experiences and patterns when making decision regarding this course. Because these patterns do not exist. Therefore we have to constantly use our 'System 2' to make decisions, weighing all the options and choosing the one best suited at that time \cite{klein}.

To choose properly, students need to make sense of the situation. If the situation made (properly) sense you could draw on your RPD to make a decision \cite{klein}. To make sense of the situation, which could be described as a crisis (because of the time pressure of delivering on friday) students use enactment. In most cases this is talking to other students to make sense of the situation and thus defuse the crisis \cite{weick}.

A side effect of these enactments between the students, next to making sense of the situation, is that the students influence each other, they make sense of the situation in more or less the same way, resulting in a homogeneous view of the situation.

%\subsection*{Differences}



\section*{3.2}
Looking back I don't believe we spend our time very wisely. We spend a lot of time on performing the interviews, specifically looking for teachers. We also spend quite some time on researching with little tangible results. We probably could have spend the time more effectively on our goals and the organisation of our data.

That said we did make explicit decisions about where we were going to spend time on and how much. Looking at the NDM (naturalistic decision making) model our decision making could be explained by first seeking more information, reading the assignments and looking at the literature to get familiar with what we had to do.

When this was more or less clear, we assessed the various actions that we had to take (perform the individual reading assignments, conduct interviews, reorganize data etc.). We then chose to perform the interviews first, starting this assignment and talking about our course of action (which could be compared to the Mental Simulation) it occurred to us that the situation was not entirely up to our expectations (our expectations were violated).

Reassessing our situations by going back to all the available assignments we came to the conclusion that we needed some information from one of the reading assignments to properly prepare for and perform the interviews.

Assessing the action of first performing the individual reading assignments yielded a positive result and this shaped our course of action.

A similar approach was then used for the other assignments.


%\clearpage
\section*{3.3}
\subsection*{Example}
The following reads as a graph definition:

\begin{itemize}
	\item[\textbf{Key}] Maintain [GoalIdentifier]:
	\emph{Goal/requirement definition.}
	
	\begin{itemize}
		\item Maintain [SubGoalIdentifier]:		
		\emph{Goal/requirement definition.}
		
		\item[\textbf{Key}] Maintain [SubGoalIdentifier]:	
		\emph{Goal/requirement definition.}
		
		\begin{itemize}
			\item Achieve [SubSubGoalIdentifier]:		
			\emph{Goal/requirement definition.}
			
			\item Achieve [SubSubGoalIdentifier]:	
			\emph{Goal/requirement definition.}
		\end{itemize}
	\end{itemize}
\end{itemize}

The \emph{key} defines the type of connector (\emph{and/or}) and the indented sub-goals contribute to their main goal. In the example above the goal with identifier \emph{GoalIdentifier} has two goals contributing to its success (both with identifier \emph{SubGoalIdentifier}, with the latter also having two sub-goals), if its key was an \emph{and} connector both of these sub-goals would have to be satisfied.\\\\ 


\subsection*{BlackBoard goals}

\begin{itemize}
	\item[\textbf{And}] Maintain [DigitalInformationPortal]:
	\emph{\textbf{If} a teacher or student wants to exchange information \textbf{then always} the system provides a means to do so.}
	
	\begin{itemize}
		\item Achieve [MakeInformationAvailable]:		
		\emph{\textbf{If} a teacher wants to make information available for its students \textbf{then sooner-or-later} he can send this.}
		
		\item Achieve [GetAvailableInformation]:
		\emph{\textbf{If} a student wants to retrieve available information \textbf{then sooner-or-later} he can do so.}
		
		\item Avoid [UnauthorizedAccess]:
		\emph{\textbf{Always not} studends can get information not (yet) meant for them.}
	\end{itemize}
\end{itemize}

\subsection*{Knowledge map}
Knowledge map on how to achieve these goals (I only list the sub goals as the composite goals will be satisfied exclusively by its sub-goals):

\begin{itemize}
	\item \textbf{MakeInformationAvailable}
	\begin{itemize}
		\item Agent: \emph{teacher}
		\item The teacher wants to make certain information available for its students (lectures, assignments, papers etc).
		\item \textbf{To achieve this goal the teacher has to be able to get files to its students.}
	\end{itemize}
	
	\item \textbf{GetAvailableInformation}
	\begin{itemize}
		\item Agent: \emph{student}
		\item The student needs to receive certain information from the teacher (lectures, assignments, papers etc).
		\item \textbf{To achieve this goal the student has to be able to get these files whenever it suits him.}
	\end{itemize}
	
	\item \textbf{UnauthorizedAccess}
	\begin{itemize}
		\item Agent: \emph{Student}
		\item Certain files need not (yet) be available for the students because the information is entirely not meant for the student or because he may not yet see it at this time (such as certain assignments or answers).
		\item Thus the system needs to differentiate between students, and who can see what and when. 
		\item \textbf{To achieve this goal students should be uniquely identifiable and the teacher must be able to choose who can view what and when.}
	\end{itemize}
\end{itemize}

Looking at this knowledge map the goal graph should be further elaborated until each goal only affects a single agent. This yields the list of requirements.

\chapter{References}

\begin{thebibliography}{9}
	
	\bibitem{klein}
	Naturalistic Decision Making, \\
	\emph{Gary Klein, Klein Associates, Division of ARA, Fairborn, Ohio.}
	
	\bibitem{weick}
	Enacted Sensemaking in Crisis Situations, \\
	\emph{Karl E. Weick, University of Michigan, USA.}
	
\end{thebibliography}


\appendix


\end{document}
