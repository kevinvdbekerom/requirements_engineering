\documentclass[11pt,a4paper,english]{uvamath}
\usepackage[english]{babel}

\usepackage{amsmath, amsfonts, amssymb, a4wide, fancyhdr, lineno, graphicx, epsfig, soul, color, hyperref}
\usepackage[square, numbers]{natbib}

% Running line numbers:
\linenumbers
% Number only every 5:th line:
\modulolinenumbers[5]

%Nodig om een bibliography midden in het artikel te zetten, ipv aan het einde zoals eigenlijk gebruikelijk is
\renewcommand{\bibsection}{}

% TODO command
\newcommand{\todo}[1]{
    \hl{#1}
}

% The things that should be filled in by each group, depending on their situation, are written in a todo command, \todo{like this text}. All text in normal the normal font, is applicable for any group. However, everyone is free to adapt any text, and it is even suggested to look at all text critically and make changes if needed.

% Project specific commands
\author{Tom van Duist \& Kevin van den Bekerom}

\newcommand{\projectname}{\todo{Project Name}\ }


\newcommand{\aanpassen}[1]{ {\sethlcolor{green} \hl{#1}} }

\title{Reading assignment week 3}
%Variables
\newcommand{\TitelAbbr}{}
\newcommand{\Version}{0.1}



\what{}
\supervisors{}
\author{Tom van Duist}


\begin{document}

\maketitle

\clearpage


\chapter*{Reading assignments week 3}

\section*{3.1}

\todo{explain differences in course assignment perception in accordance with the Weick and Klein papers}


\section*{3.2}
Looking back I don't believe we spend our time very wisely. We spend a lot of time on performing the interviews, specifically looking for teachers. We also spend quite some time on researching with little tangible results. We probably could have spend the time more effectively on our goals and the organisation of our data.

That said we did make explicit decisions about where we were going to spend time on and how much. Looking at the NDM (naturalistic decision making) model our decision making could be explained by first seeking more information, reading the assignments and looking at the literature to get familiar with what we had to do.

When this was more or less clear, we assessed the various actions that we had to take (perform the individual reading assignments, conduct interviews, reorganize data etc.). We then chose to perform the interviews first, starting this assignment and talking about our course of action (which could be compared to the Mental Simulation) it occurred to us that the situation was not entirely up to our expectations (our expectations were violated).

Reassessing our situations by going back to all the available assignments we came to the conclusion that we needed some information from one of the reading assignments to properly prepare for and perform the interviews.

Assessing the action of first performing the individual reading assignments yielded a positive result and this shaped our course of action.

A similar approach was then used for the other assignments.

\clearpage
\section*{3.3}
\subsection*{Example}
The following reads as a graph definition:

\begin{itemize}
	\item[\textbf{Key}] Maintain [GoalIdentifier]:
	\emph{Goal/requirement definition.}
	
	\begin{itemize}
		\item Maintain [SubGoalIdentifier]:		
		\emph{Goal/requirement definition.}
		
		\item[\textbf{Key}] Maintain [SubGoalIdentifier]:	
		\emph{Goal/requirement definition.}
		
		\begin{itemize}
			\item Maintain [SubSubGoalIdentifier]:		
			\emph{Goal/requirement definition.}
			
			\item Maintain [SubSubGoalIdentifier]:	
			\emph{Goal/requirement definition.}
		\end{itemize}
	\end{itemize}
\end{itemize}

The \emph{key} defines the type of connector (\emph{and/or}) and the indented sub-goals contribute to their main goal. In the example above the goal with identifier \emph{GoalIdentifier} has two goals contributing to its success (both with identifier \emph{SubGoalIdentifier}, with the latter also having two sub-goals), if its key was an \emph{and} connector both of these sub-goals would have to be satisfied.\\\\ 


\subsection*{BlackBoard goals}

\begin{itemize}
	\item[\textbf{And}] Maintain [DigitalInformationPortal]:
	\emph{\textbf{If} a teacher or student wants to exchange information \textbf{then always} the system provides a means to do so.}
	
	\begin{itemize}
		\item Maintain [MakeInformationAvailable]:		
		\emph{\textbf{Always} a teacher can make information available for its students.}
		
		\item Maintain [GetAvailableInformation]:
		\emph{\textbf{Always} a student can get available information.}
		
		\item Avoid [UnauthorizedAccess]:
		\emph{\textbf{Always not} studends can get information not (yet) meant for them.}
	\end{itemize}
\end{itemize}


\chapter{References}

\begin{thebibliography}{9}
	
\end{thebibliography}


\appendix


\end{document}
